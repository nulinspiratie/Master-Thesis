%!TEX root = ../thesis.tex
%!TEX root = ../thesis.tex

\newcommand{\vrms}{\text{v}_\text{n}^\text{rms}}
\newcommand{\vsq}{\overline{\text{v}_\text{n}^2}}
\newcommand{\vn}{\text{v}_\text{n}}
\newcommand{\vin}{\text{v}_\text{in}}
\newcommand{\vout}{\text{v}_\text{out}}
\newcommand{\Sin}{\text{S}_{\vn}^\text{in}}
\newcommand{\Sout}{\text{S}_{\vn}^\text{out}}
\newcommand{\Svn}{\text{S}_{\vn}}


\chapter{Noise characterization}
\label{ch:noise}
% TODO: Left to find out:
% Is the power to V_in conversion correct?

When performing measurements one is faced with the reality that no component is ideal. When a signal passes through the different parts of a set-up, noise is constantly being added. Noise is the term given for all the random fluctuations that are added to the signal. These fluctuations are the cumulative result of several noise source contributions.

Thermal noise is one of the most common sources of noise. It is the result of the random thermal fluctuations of electrons. It is an example of frequency-independent noise, also known as white noise. Noise sources can also be frequency-dependent, such as TLS, as discussed in Chapter~\ref{part:DRIE}. This is an example of $1/f$-noise: the amount of noise added increases with decreasing frequency. In fact, truly frequency-independent noise does not exist, as even white noise has been observed to decrease at extremely high frequencies (\SI{\sim e15}{\hertz}). At these frequencies a quantum correction needs to be added \cite[p50]{vasilescu2006electronic}.

When performing measurements one important question to ask is how much noise is being contributed to the signal. In this chapter a model is presented for the general set-up used for measuring superconducting resonators and qubits. Using this model it is possible to characterize the amount of noise by determining its associated noise temperature. Finally, this model is applied to the set-up used to characterize the resonators presented in Chapter~\ref{part:DRIE}.

\section{Characterizing noise}

\subsection{Circuit representations}


\begin{figure}[h]
    \centering
    \begin{subfigure}[b]{.43\textwidth}
        \includegraphics[width=\textwidth]{Figures/Noise/thevenin.png}%\figureinset{(a)}{2.55}{1.5}
        \caption{Th\'evenin representation}
        \label{fig:thevenin}
    \end{subfigure}
    \begin{subfigure}[b]{.49\textwidth}
        \includegraphics[width=\textwidth]{Figures/Noise/norton.png}%\figureinset{(b)}{2.55}{1.5}
        \caption{Norton representation}
        \label{fig:norton}
    \end{subfigure}
    \caption{Two equivalent representations of a system containing noise. Panel~\textbf{(a)} shows the Th\'evenin representation, in which a noiseless voltage source is connected in series with a noise voltage source. Panel~\textbf{(b)} shows the Norton representation, in which a noiseless current source is connected in parallel with a noise current source.}
\end{figure}

There are two circuit representations in which we can depict a system with a noise contribution: The Th\'evenin representation, and the Norton representation. In the Th\'evenin representation we can model the system as a voltage source, and the noise added to the system is a noise voltage source connected in series. In the Norton representation the system is a current source and the noise added is a noise current source connected in parallel to the current source. These two representations are identical and can be converted to each other. In this section we will adopt the Thev\'enin representation, and so the signal will be a voltage source combined in series with a noise voltage source.

Assuming the signal to be at a fixed frequency $\omega$ and amplitude $A$, the combined voltage $v(t)$ is then given by:

\begin{equation}
    \text{v}(t) = \text{v}_\text{signal}(t) + \text{v}_\text{noise}(t) = A \cos{\omega t} + \vn(t)
\end{equation}

Note that the mean value of the noise voltage $\overline{\vn}$ is equal to zero. The amount of noise can be quantified by the root-mean square noise voltage $\vrms$:
\begin{equation}
    \vrms = \sqrt{\overline{\text{v}^2} - \overline{\text{v}}^2} = \sqrt{\vsq}
    \label{eqn:v_rms}
\end{equation}

\subsection{Noise power spectral density}

One way of quantifying the noise of the system is through the noise power spectral density $S(f)$, which is the distribution of noise power per unit bandwidth as a function of frequency. For the Th\'evenin representation, the noise spectral density is defined in terms of voltage. When the only noise in the system is white noise, the spectral density is independent of frequency. It is then given by:

\begin{equation}
    S = \frac{\vsq}{\Delta f}
    \tagaddtext{[\si[per-mode=symbol]{\volt \squared \per \hertz}]}
    \label{eqn:noise spectral density definition}
\end{equation}

In this equation $\Delta f$ is the noise bandwidth. This is the bandwidth over which the noise is measured.

\newpage
\section{The model}

\begin{figure}
    \centering
    \includegraphics[width=.8\textwidth]{Figures/Noise/Noise model.png}
    \caption{Schematic representation of the measurement set-up including a noise source.}
    \label{fig:noise model}
\end{figure}


As shown in the schematic in Figure~\ref{fig:noise model}, we can model our set-up as a combination of four elements:

\begin{enumerate}
    \item A noise source.
    \item An amplifier to amplify the weak signal exiting the fridge.
    \item A mixer to downconvert the signal.
    \item A low-pass filter to remove unwanted high-frequency signal.
\end{enumerate}



\subsection{Noise source}

Using the Th\'evenin model, we can approximate the components up to the first amplifier in the fridge as a voltage source, with a noise voltage source connected in series.

We can include the noise added by the amplifier in the noise voltage source, in which case we assume the amplifier to be ideal. Furthermore, we assume the signal to be amplified sufficiently, such that the mixer and low-pass filter add a negligible amount of noise. We also ignore effect such as mixer leakage. With these assumptions all of the noise is originated from the noise voltage source.

We can associate an effective noise temperature to the noise voltage source. The noise temperature is defined as the temperature at which a resistor would produce an equal amount of noise. Note that, since we are comparing the system to a resistor, the noise needs to have an (approximately) white spectrum.

According to Nyquist's theorem \cite[p47]{vasilescu2006electronic}, if the system experiences white noise, and is in thermal equilibrium, the root-mean square noise voltage $\vrms$ is given by:

\begin{equation}
    \vrms = \sqrt{\vsq} = 4 k_B T R \Delta f
    \label{eqn:noise rms voltage}
\end{equation}

In this equation $\Delta f$ is the bandwidth over which the noise is integrated, $k_B$ is the Boltzmann constant, $T$ is the noise temperature of the noise source, and $R$ is the impedance of the system. We see that the noise added depends linearly on the bandwidth over which is integrated.

Combining Equations~\ref{eqn:noise spectral density definition} and \ref{eqn:noise rms voltage}, the noise power spectral density $S_{\vn}$ can be rewritten as:

\begin{equation}
S = \frac{\vsq}{\Delta f} = 4 k_B T R
\end{equation}





\subsection{Amplification}
During the amplification stage both the signal and the noise is amplified by the same amount. This amount of amplification is determined by the gain $G$, which is defined as the ratio between the output voltage and the input voltage:

\begin{equation}
    G = \frac{\text{v}_\text{out}}{\text{v}_\text{in}}
    \label{eqn:gain}
\end{equation}

According to the maximum power transfer theorem, the maximum power transfer between a source and load occurs when the impedances of source and load are matched, in which case half of the power is transferred. From this it follows that in the amplification process half of the signal is dissipated. However, as $G$ is defined as the ratio between $\text{v}_\text{out}$ and $\text{v}_\text{in}$ (Equation~\ref{eqn:gain}), the factor $\frac{1}{2}$ is included in $G$.

During amplification not only the signal is amplified with gain G: the noise is amplified by the same amount. Defining the noise power spectral density before amplification as $S^\text{in}$, and after amplification as $S^\text{out}$, the following relation holds:

\begin{equation}
    S^\text{out} = G^2\; S^\text{in} = G^2 4 k_B T R
    \label{eqn:noise power spectral density amplification}
\end{equation}

Note that in Equation~\ref{eqn:noise power spectral density amplification}, the gain $G$ is squared. This is due to the fact that the noise power spectral density depends quadratically on the root-mean square noise voltage (Equation~\ref{eqn:noise spectral density definition}).

In our actual set-up the amplification occurs in multiple stages. Aside from amplifying the signal and its noise, at each stage additional noise, originating from the amplifier itself, is added as well. This added noise is then also amplified in the next amplification stage.   Therefore it is always best to have the amplifier with highest gain and lowest noise temperature as the first amplifier in the chain. For more information see Friis formula \cite[p103]{vasilescu2006electronic}.



\subsection{Downconversion}

After amplification the frequency $\omega$ of the signal is still in the GHz range. In homodyne or heterodyne detection the signal is downconverted to DC (homodyne) or to a lower frequency (heterodyne), such that it can be measured more easily. To downconvert the signal, it is mixed in a mixer with a local oscillator (LO) signal having the same frequency $\omega$ (homodyne) or a slightly higher frequency $\omega + \Delta \omega$ (heterodyne). The mixer effectively multiplies the two signals. If the signal exiting the amplifier is given by $\text{v}(t) = A\cos \omega t$, then, ignoring a possible phase difference, the signal at the output of the mixer is given by:

\begin{align}
    \text{v}(t) \cdot \cos{\omega t}& = A \cos{\omega t} \cdot \cos{(\omega + \Delta \omega ) t} \notag\\
        & = \frac{1}{2} A \left[\cos{(2\omega + \Delta \omega)t} + \cos{\Delta \omega t}\right]
        \label{eqn:mixer}
\end{align}

As can be seen from Equation~\ref{eqn:mixer}, the output signal contains both the sum and the difference of the two signals. However, as the sum of both frequencies is in the GHz range, it can be filtered out using a low-pass filter, leaving only the downconverted signal, which is the result of the difference between the two frequencies. Note that the amplitude of the signal is reduced by a factor two. The noise amplitude is also reduced by a factor 2. Furthermore, in the case of a homodyne set-up, the difference signal is simply a DC signal ($\Delta \omega = 0$), while in the case of heterodyne the signal still contains a slow frequency $\Delta \omega$. For simplification we assume our set-up to be a homodyne set-up, although the result is similar in the case of a heterodyne set-up.

\subsection{Low-pass filtering}

In the case of homodyne detection the signal at the frequency of interest is downconverted to DC. However, the signal at other frequencies have not disappeared; in the mixer these have also shifted in frequency. Since the signal of interest is at DC, a low-pass filter can be used to filter out signal above a certain frequency.

The frequency above which a low-pass filter will filter out the signal is defined by its cut-off frequency $f_\text{c}$. The cut-off frequency $f_\text{c}$ is the frequency at which the signal is attenuated by \SI{3}{\decibel}. For first-order low-pass filters the noise bandwidth $\Delta f$ is related to the filter cut-off frequency $f_\text{c}$ by \cite[p81]{vasilescu2006electronic}:

\begin{equation}
    \Delta f = \frac{\pi}{2} f_\text{c}
    \label{eqn:noisebandwidth}
\end{equation}

\section{Noise temperature}

In the previous sections the influence of each of the components on the signal and on the noise has been analyzed. Using this information it is possible to determine the signal-to-noise ratio (SNR), which is the ratio between the average power of the signal and the average power of the noise. The SNR is a measure for how well a signal can be separated from the noise, and is given by:

\begin{align}
    \text{SNR} = \frac{\overline{\vout}^2}{\vsq} = & \frac{1/4 \;G^2 \; \overline{\vin}^2}{ \Sout \; \Delta f}\notag\\
        = & \frac{1/4 \; G^2 \; \overline{\vin}^2}{G^2 \; \Sin \;\frac{\pi}{2}\;f_\text{c}}\notag\\
        = & \frac{\overline{\vin}^2}{2 \; \pi \; k_B \; T \; R \; f_\text{c}}
    \label{eqn:SNR}
\end{align}

% TODO: factor 1/4 for both signal and noise?
Note that the factor $1/4$ is because the amplitude is lowered by a factor of $2$ due to mixing. Equation~\ref{eqn:SNR} can be rewritten such that we have an expression for the noise temperature of the system:

\begin{equation}
    T =\frac{\overline{\vin}^2}{2 \; \pi \; k_B \; R \; f_\text{c} \; \text{SNR}}
    \label{eqn:noise temperature}
\end{equation}


\section{Results}
\label{sec:noise_results}
\begin{figure}[h]
    \centering
    \includegraphics[width = .7 \linewidth]{Figures/Noise/noise_temperature_versus_power.png}
    \caption{Noise temperature versus frequency. The noise temperature has been calculated for $160$ frequencies in the range \SIrange{1}{9}{\giga \hertz}. For each frequency $2001$ points were measured, from which the signal, noise, SNR, and noise temperature was determined. Measurements were performed at an input power of \SI{-113}{\dBm} and an IF bandwidth of $\Delta f = $\SI{300}{\hertz}.}
    \label{fig:Noise temperature}
\end{figure}

Using the Rhode \& Schwarz ZVM vector network analyzer, The transmission has been measured for $160$ equidistant frequencies in the range \SIrange{1}{9}{\giga \hertz}. For each frequency a total of $2001$ points was measured with an IF bandwidth $\Delta f = $ \SI{300}{\hertz}. From these measurements the signal-to-noise ratio has been determined for each frequency. With knowledge of the SNR, the noise temperature has then been determined as a function of frequency using Equation~\ref{eqn:noise temperature} . The result is shown in Figure~\ref{fig:Noise temperature}.

From Figure~\ref{fig:Noise temperature} it is clear that the noise temperature is highly temperature-dependent. In the frequency range \SIrange{4}{8}{\giga \hertz} the noise temperature is quite low, never reaching above \SI{10}{\kelvin}. This is exactly the bandwidth of the cryogenic low-noise amplifier by Low Noise Factory, which is the first amplifier in the amplification chain. From the specifications of the amplifier, the noise temperature of the amplifier has been calculated at an ambient temperature of \SI{8}{\kelvin}, and equals roughly \SI{4}{\kelvin} for the entire bandwidth. Comparing the amplifier specifications with Figure~\ref{fig:Noise temperature}, it is likely that in the frequency range \SIrange{4}{8}{\giga \hertz} the first amplifier is the component contributing most to the total noise temperature.

Outside the \SIrange{4}{8}{\giga \hertz} frequency band, however, the noise temperature rapidly increases. This is partly due to the frequency lying outside of the bandwidth of the amplifier, in which case the amplification will be lower. However, this is not an adequate explanation for the fact that the noise temperature increases to several hundred thousand Kelvin. The reason for this unrealistic noise temperature is that in our model we did not take into account the noise added by components after the amplifier. While the gain of the amplifier decreases outside its bandwidth, the components after the amplification will still add the same amount of noise. When the gain decreases by a significant amount, the relative contribution of these post-amplification noise sources will increase. Furthermore the assumption that the noise spectrum is white is no longer correct at low frequencies, where $1/f$ noise starts to contribute.

\begin{figure}[h]
    \centering
    \includegraphics[width = .6\linewidth]{Figures/Noise/noise_temperature_versus_power_detailed.png}
    \caption{Detailed scan of noise temperature versus frequency in the range \SIrange{2.6}{2.9}{\giga \hertz}. The resonator with $f_0 = $\SI{2.75}{\giga \hertz} (red dot) seems to reside at a local minimum of the noise temperature.}
    \label{fig:noise temperature 2GHz}
\end{figure}

From Figure~\ref{fig:Noise temperature} it can be seen that the noise temperature can vary by a large amount between consecutive points. To determine whether this variation is due to a large uncertainty, or due to the noise temperature actually fluctuating strongly with varying frequency, a more detailed scan has been performed in the frequency region \SIrange{2.6}{2.9}{\giga \hertz}, in which one resonator has a resonance frequency. The result is shown in Figure~\ref{fig:noise temperature 2GHz}. As the curve of the detailed scan follows the curve of the coarse scan pretty closely, it can be concluded that the noise temperature of the system in fact fluctuates quite strongly with varying frequency.

Another point of interest is that the resonance frequency of the resonator lies near the local minimum of the noise temperature in that region. This is quite a stroke of luck, as a slightly higher or lower frequency would have resulted in a much higher noise temperature.



\section{Conclusion and future work}

The noise temperature gives us an estimate of the noise added to the system. It has been shown that the noise temperature can fluctuate strongly with varying frequency. In the model used to estimate the noise temperature it has been shown that outside the bandwidth of the amplifier the model breaks down. At this point the noise added by the components after the amplification, and indeed even the amplifiers themselves, needs to be taken into account to obtain accurate estimates of the noise temperature.

However, even outside of the bandwidth of the amplifier, there are frequency regions in which the noise temperature may still be acceptable. It is therefore a good idea to initially perform measurements of the noise temperature of the set-up. This will give an indication of the signal-to-noise ratio, from which accurate estimates can be made as to what the amount of measurement time is needed to obtain a desired SNR.

The noise temperature measurements were performed using the Rhode \& Schwarz ZVM vector network analyzer. It would be interesting to see how other measurement set-ups would compare to the vector network analyzer. One interesting candidate would be a heterodyne detector. However, as the vector network analyzer can also measure phase, a fair comparison would also require the heterodyne detector to be able to measure the phase. This heterodyne detector is currently being set up, and will hopefully soon yield interesting results.

Aside from only comparing the noise temperature, other properties are also important when comparing two set-ups. One of these is the duty cycle, which is the percentage of time acually spent measuring. For the vector network analyser the duty cycle seems to be around $50\%$, provided that a single measurement sweep takes at least a few seconds. Other set-ups may therefore offer an improvement in the duty cycle. Furthermore, properties such as phase stability and uncertainty would also be interesting to compare.

Another interesting measurement would be to see if the noise temperature as a function of frequency remains the same in future cooldowns, and for different samples.



\externaldocument[A-]{Chapters/DRIE.tex}
\externaldocument[A-]{Chapters/Muxmon.tex}

\chapter{Photon number calculation}
  \label{ch:photon number calculation}

  In cQED resonators usually operate in the single-photon regime. It is therefore important to know what power is required to reach this regime. In the equilibrium state of the resonator the power absorbed $P_\text{abs}$ by the resonator is equal to the power leaking out of the resonator, and so the absorbed power $P_\text{abs}$ can be expressed as

  \begin{equation}
    P_\text{abs} = \left<n_\text{ph}\right>\hbar\omega_r \kappa,
    \label{eq:Pabs-nph}
  \end{equation}
  where $\kappa=\omega_r/Q_i$ is the cavity decay rate.

  The absorbed power $P_\text{abs}$ can also be expressed in terms of the input power $P_\text{in}$ entering the feedline. We note that the absorbed power $P_\text{abs}$ is equal to the input power $P_\text{in}$ entering the feedline minus the reflected power $P_\text{refl}$ and the transmitted power $P_\text{trans}$. These are given by

  \begin{align}
    P_\text{refl} = P_\text{in}\left|S_{11}\right|^2;\\
    P_\text{trans} = P_\text{in}\left|S_{21}\right|^2,
  \end{align}
  where $S_{11}$ and $S_{21}$ are the scattering parameters. At the resonance frequency $S_{21}=\frac{Q_c}{Q_i+Q_c}$ (see Eq.~\ref{eq:S21min}), and $S_{11}=S_{21}-1$. Therefore the absorbed power is equal to

  \begin{equation}
    P_\text{abs}=1-P_\text{refl}-P_\text{trans}=\frac{2Q_l^2}{Q_c Q_i}P_\text{in}.
    \label{Pabs-Pin}
  \end{equation}
  Combining Eqs.~\ref{eq:Pabs-nph} and~\ref{Pabs-Pin}, we arrive at a formula for the mean photon number $\left<n_\text{ph}\right>$:

  \begin{equation}
    \left<n_\text{ph}\right> = \frac{2}{\hbar \omega_r}\frac{Q_l^2}{Q_c}P_\text{in}.
  \end{equation}
  The input power $P_\text{in}$ can be found by measuring the transmission of the components between the generator and feedline.



\chapter{Duplexer isolation}
  \label{ch:Duplexer isolation}

  \begin{figure}[h]
    \centering
    \includegraphics[width=\textwidth]{Figures/Appendix/Duplexer isolation.png}
    \caption{Isolation of the Duplexer. Legend indices correspond to switch state ($1$ means open, $0$ means closed, lsb corresponds to bottom drive line, msb to top drive line). Transmission is measured relative to the transmission when all channels are open (11). The black line corresponds to the measured noise floor. The red-dashed line indicates the frequency of the top and bottom qubit, and the blue-dashed line to the ancilla qubit, during the Muxmon experiment.}
    \label{fig:Duplexer isolation}
  \end{figure}

  The isolation of the Duplexer switches has been measured. During the measurements a signal was sent through input channel $1$ and $2$, corresponding to the signal of the main Gaussian pulse, and derivative pulse, respectively. For each input channel the transmission to each of the two output channels has been measured. These output channels are connected to the drive lines of the top and bottom qubit during the Muxmon experiment. For each input-output-port combination the transmission has been measured for different switch configurations, which were controlled using the nanosecond switches. Four different switch combinations were used, corresponding the the four distinct possibilities of the signal being directed to each of the two output channels.

  The results of the isolation measurements are shown in Fig.~\ref{fig:Duplexer isolation}. As can be seen in all four input-output combinations the isolation is around \SI{50}{\dBm} at the frequency of top and bottom qubit. Furthermore, the isolation does not seem to depend strongly on whether the switch to the other output port is open or closed.

\chapter{Chip characterization}

  \section{Muxmon1 cross-driving}
    \label{sec:Muxmon1 cross-driving}
      \begin{table}[h]
        \begin{tabular}{l c c}
          \toprule
          qubit & \multicolumn{2}{c}{cross-driving (\%)} \\
          \cmidrule(lr){2-3}
               & top-ancilla drive line & bottom-ancilla drive line \\
          \midrule
          Top     & $100$    & $5.22$ \\
          Ancilla & $75$ & $62.5$ \\
          Bottom  & $1.05$ & $100$    \\
          \bottomrule
        \end{tabular}
        \label{tab:Muxmon1 cross-driving}
      \end{table}

  \section{Muxmon0 resonator buses}
    \label{sec:Resonator buses}

    \begin{figure}[h]
      \centering
      \includegraphics[width=\textwidth]{Figures/Appendix/resonator buses.png}
      \caption{Normalized transmission showing the resonator buses of the top and bottom qubit, at frequencies \SI{4.88}{\giga \hertz} and \SI{4.97}{\giga \hertz}, respectively.}
      \label{fig:resonator buses}
    \end{figure}

    The top and bottom qubit are each coupled to a resonator bus. When the frequency of a qubit approaches that of the bus, they experience an avoided crossing. In Fig.~\ref{fig:resonator buses} these avoided crossing are shown for the top and bottom qubit. For the top qubit the frequency of the resonator bus is found to be \SI{4.88}{\giga \hertz}. For the bottom qubit the signal was considerably worse, but the frequency of the resonator bus is found to be roughly \SI{4.97}{\giga \hertz}. For the top qubit the coupling strength $g$ between the qubit and the bus can be extracted. It is equal to half the minimum distance, and approximately equal to \SI{27}{\mega \hertz}.

\chapter{Additional notes}
  \section{Qubit characterization}
      \subsection{Finding the qubit sweet spot using a one-dimensional scan}
        \label{Finding the qubit sweet spot using a one-dimensional scan}
        \begin{figure}[tb]
          \centering
          \includegraphics[width=.6\linewidth]{Figures/Qubit characterization/Resonator vs DAC linecut.png}
          \caption{Fixed-frequency transmission versus DAC current. The frequency is chosen to be slightly below the resonator frequency, found when the DAC current is set to zero. The symmetry point in the linecut corresponds to the sweet spot of the ancilla qubit.}
          \label{fig:resonator vs dac linecut}
        \end{figure}

        As explained in Sec.~\ref{Scan for qubit sweet spots}, the sweet spot of a qubit can be found by performing a series of resonator scans while varying the DAC current. There is, however, a faster approach for finding the qubit sweet spot, at the cost of providing less information. This is done by choosing a fixed frequency close to the resonator  frequency $\wres$ (preferably slightly below, where the transmission slope is steepest). By measuring the amount of transmission as the DAC voltage is being varied, one essentially obtains a linecut of the 2D scan. The idea this measurement is that if the qubit frequency $\fqub$ decreases, so does resonator frequency, resulting in a decrease in transmission (closer to $\wres$). Likewise, if the qubit frequency $\wqub$ increases, so does the resonator frequency, resulting in an increase in transmission (further away from $\wres$).

        At the qubit's sweet-spot, the resonator's frequency $\wres$ is at a maximum, and so the transmission should also be at a maximum. Furthermore, because the qubit frequency is symmetric with respect to its sweet spot, so should the transmission. The sweet spot can therefore be determined by finding the symmetric point in the transmission. In Fig.~\ref{fig:resonator vs dac linecut} one such linecut is shown. The transmission is clearly symmetric, and the symmetric point is equal to the qubit sweet spot.

        If the resonator's frequency $\wres$ shifts by a large amount in the course of this measurement, it becomes harder to determine where the sweet spot is (although even then often it can still be discerned). Nevertheless, this method is considerably faster than performing a full two-dimensional scan of frequency versus DAC voltage, and in most cases does provide sufficient information to determine the sweet spot.


      \subsubsection{Spectroscopy}
        \begin{itemize}
          \item If the deviation in transmission becomes less due to more detuning, increasing the power can also increase the contrast.
        \end{itemize}
      \subsubsection{Flux matrix}
        After determining the flux matrix $\boldsymbol{F}$, there will still be some small remaining cross coupling, which depends on the accuracy of the measurements. The process of creating a flux matrix can then be repeated, but instead of using DAC voltages as the varying parameters to construct matrix $\boldsymbol{M}$, the virtual fluxes should be used. Furthermore, as the cross-coupling is small compared to before, the flux range can be much greater, such that small slopes can be accurately measured. The resulting flux matrix $\boldsymbol{F}_2$ can then simply be multiplied with the first flux matrix $\boldsymbol{F}$ to obtain a more accurate final flux matrix.
    \chapter{AllXY pulse sequence}
      \label{ch:AllXY pulse table}

      \begin{table}[h]
        \centering
        \begin{tabular}{c c c c }
          \toprule
          number & ideal $\left<z\right>$ & First pulse & Second pulse \\
          \midrule
          1  & 1  & $I$         & $I$         \\
          2  & 1  & $X_\pi$     & $X_\pi$     \\
          3  & 1  & $Y_\pi$     & $Y_\pi$     \\
          4  & 1  & $X_\pi$     & $Y_\pi$     \\
          5  & 1  & $Y_\pi$     & $X_\pi$     \\
          6  & 0  & $X_{\pi/2}$ & $I$         \\
          7  & 0  & $Y_{\pi/2}$ & $I$         \\
          8  & 0  & $X_{\pi/2}$ & $Y_{\pi/2}$ \\
          9  & 0  & $Y_{\pi/2}$ & $X_{\pi/2}$ \\
          10 & 0  & $X_{\pi/2}$ & $Y_\pi$     \\
          11 & 0  & $Y_{\pi/2}$ & $X_\pi$     \\
          12 & 0  & $Y_\pi$     & $Y_{\pi/2}$ \\
          13 & 0  & $X_\pi$     & $X_{\pi/2}$ \\
          14 & 0  & $X_{\pi/2}$ & $X_\pi$     \\
          15 & 0  & $X_\pi$     & $X_{\pi/2}$ \\
          16 & 0  & $Y_{\pi/2}$ & $Y_\pi$     \\
          17 & 0  & $Y_\pi$     & $Y_{\pi/2}$ \\
          18 & -1 & $X_\pi$     & $I$         \\
          19 & -1 & $Y_\pi$     & $I$         \\
          20 & -1 & $X_{\pi/2}$ & $X_{\pi/2}$ \\
          21 & -1 & $Y_{\pi/2}$ & $Y_{\pi/2}$ \\
          \bottomrule
        \end{tabular}
        \caption{The 21 pulses that comprises the AllXY pulse sequence.}
        \label{tab:AllXY sequence}
      \end{table}

      In Table~\ref{tab:AllXY sequence} the 21 pulse combinations are shown that comprise the AllXY pulse sequence. The AllXY measurement is able to diagnose specific sources contributing to gate errors (see Sec.~\ref{ssec:AllXY}). For a more detailed analysis on the AllXY see Reed's thesis~\cite{Reed}.

  \chapter{Randomized benchmarking}
    \section{Clifford gate decomposition}
      \label{ssec:Clifford gate decomposition}


        \begin{tabular}{c c c }
          \toprule
          Clifford ID     & gate decomposition & \begin{tabular}{@{}c@{}}5 primitives decomposition \\\noalign{\smallskip} $X_{\pi/2} \quad Y_{\pi/2} \quad X_{\pi/2} \quad X_{-\pi} \quad Y_{-\pi}$ \end{tabular}\\
          \midrule
          1 & $ I$ & 0\quad\quad\;0\quad\quad\;0\quad\quad\;0\quad\quad\;0 \\
          2 & $ Y_{\pi/2} -  X_{\pi/2}$ & 0\quad\quad\;1\quad\quad\;1\quad\quad\;0\quad\quad\;0 \\
          3 & $ X_{-\pi/2} -  Y_{-\pi/2}$ & 1\quad\quad\;1\quad\quad\;0\quad\quad\;1\quad\quad\;0 \\
          4 & $ X_\pi$ & 0\quad\quad\;0\quad\quad\;0\quad\quad\;1\quad\quad\;0 \\
          5 & $ Y_{-\pi/2} -  X_{-\pi/2}$ & 0\quad\quad\;1\quad\quad\;1\quad\quad\;0\quad\quad\;1 \\
          6 & $ X_{\pi/2} -  Y_{-\pi/2}$ & 1\quad\quad\;1\quad\quad\;0\quad\quad\;0\quad\quad\;1 \\
          7 & $ Y_\pi$ & 0\quad\quad\;0\quad\quad\;0\quad\quad\;0\quad\quad\;1 \\
          8 & $ Y_{-\pi/2} -  X_{\pi/2}$ & 0\quad\quad\;1\quad\quad\;1\quad\quad\;1\quad\quad\;1 \\
          9 & $ X_{\pi/2} -  Y_{\pi/2}$ & 1\quad\quad\;1\quad\quad\;0\quad\quad\;0\quad\quad\;0 \\
          10 & $ X_\pi -  Y_\pi$ & 0\quad\quad\;0\quad\quad\;0\quad\quad\;1\quad\quad\;1 \\
          11 & $ Y_{\pi/2} -  X_{-\pi/2}$ & 0\quad\quad\;1\quad\quad\;1\quad\quad\;1\quad\quad\;0 \\
          12 & $ X_{-\pi/2} -  Y_{\pi/2}$ & 1\quad\quad\;1\quad\quad\;0\quad\quad\;1\quad\quad\;1 \\
          13 & $ Y_{\pi/2} -  X_\pi$ & 0\quad\quad\;1\quad\quad\;0\quad\quad\;1\quad\quad\;0 \\
          14 & $ X_{-\pi/2}$ & 0\quad\quad\;0\quad\quad\;1\quad\quad\;1\quad\quad\;0 \\
          15 & $ X_{\pi/2} -  Y_{-\pi/2} -  X_{-\pi/2}$ & 1\quad\quad\;1\quad\quad\;1\quad\quad\;0\quad\quad\;1 \\
          16 & $ Y_{-\pi/2}$ & 0\quad\quad\;1\quad\quad\;0\quad\quad\;0\quad\quad\;1 \\
          17 & $ X_{\pi/2}$ & 0\quad\quad\;0\quad\quad\;1\quad\quad\;0\quad\quad\;0 \\
          18 & $ X_{\pi/2} -  Y_{\pi/2} -  X_{\pi/2}$ & 1\quad\quad\;1\quad\quad\;1\quad\quad\;0\quad\quad\;0 \\
          19 & $ Y_{-\pi/2} -  X_\pi$ & 0\quad\quad\;1\quad\quad\;0\quad\quad\;1\quad\quad\;1 \\
          20 & $ X_{\pi/2} -  Y_\pi$ & 1\quad\quad\;0\quad\quad\;0\quad\quad\;0\quad\quad\;1 \\
          21 & $ X_{\pi/2} -  Y_{-\pi/2} -  X_{\pi/2}$ & 1\quad\quad\;1\quad\quad\;1\quad\quad\;1\quad\quad\;1 \\
          22 & $ Y_{\pi/2}$ & 0\quad\quad\;1\quad\quad\;0\quad\quad\;0\quad\quad\;0 \\
          23 & $ X_{-\pi/2} -  Y_\pi$ & 1\quad\quad\;0\quad\quad\;0\quad\quad\;1\quad\quad\;1 \\
          24 & $ X_{\pi/2} -  Y_{\pi/2} -  X_{-\pi/2}$ & 1\quad\quad\;1\quad\quad\;1\quad\quad\;1\quad\quad\;0 \\
          \bottomrule
        \end{tabular}

      % \begin{figure}[tb]
      %   \centering
      %   \includegraphics[width=\textwidth]{Figures/Clifford decomposition.png}
      %   \caption{Decomposition of all $24$ Cliffords into X and Y rotations}
      %   \label{fig:Clifford decomposition}
      % \end{figure}
      % \textbf{TODO:} Cite


    \section{Determining population in three states}
      \label{ssec:Determining population in three states}
      If there were no leakage present during randomized benchmarking, the full information about the state populations can be extracted from the randomized benchmarking results. However, if leakage to the second-excited state is present, the full information about the populations of the three states can be extracted using two versions of randomized benchmarking, one without a final pi pulse, and one with a final pi pulse. The final pi pulse swaps the populations in the ground and excited state. We assume that the population of the second excited-state remains unaffected by this single final pulse.

      Using these two randomized benchmarking sequences, two different signals $S_0$ and $S_1$ are measured, corresponding to a measurement without a final pi pulse, and a measurement with a final pi pulse, respectively. This leads to the following three equations:

      \begin{align}
        p_0 V_0 + p_1 V_1 + p_2 V_2 = & S_0; \notag\\
        p_1 V_1 + p_0 V_1 + p_2 V_2 = & S_1; \notag\\
        p_0 + p_1 + p_2 = &   1,
        \label{eq:three populations equations App}
      \end{align}
      where $p_i$ corresponds to the final population in state $\ket{i}$, and $V_i$ is the signal of state $\ket{i}$. Filling in $p_2 = 1 - p_0 - p_1$ into the first two equations of \ref{eq:three populations equations App}, we are left with the following set of equations:

      \begin{align}
        \begin{bmatrix}
          V_0 - V_2 & V_1 - V_2 \\
          V_1 - V_2 & V_0 - V_2
        \end{bmatrix}
        \begin{bmatrix}
          p_0 \\
          p_1
        \end{bmatrix}
        =
        \begin{bmatrix}
          S_0 - V_2 \\
          S_1 - V_2
        \end{bmatrix}.
      \end{align}
      This set of equations can be easily solved by matrix inversion, resulting in the following three populations:

      \begin{align}
        \begin{bmatrix}
          p_0 \\
          p_1
        \end{bmatrix}
        = &
        \left((V_0-V_2)^2 - (V_1 - V_2)^2\right)^{-1}
        \begin{bmatrix}
          V_0 - V_2 & -V_1 + V_2 \\
          -V_1 + V_2 & V_0 + V_2
        \end{bmatrix}
        \begin{bmatrix}
          S_0 - V_2 \\
          S_1 - V_2
        \end{bmatrix}; \notag\\
        p_2 = & 1 - p_0 - p_1.
        \label{eq:RB populations using final pi}
      \end{align}
      If one has knowledge of all thee signals $V_0$, $V_1$ and $V_2$ (see Sec.~\ref{ssec:Second excited-state} for information on measuring $V_2$), the three populations can be obtained using Eq.~\ref{eq:RB populations using final pi}.
      \newpage
  \section{Two-qubit randomized benchmarking leakage}
    \label{sec:Two-qubit randomized benchmarking leakage}
        \begin{table}[h]
          \begin{tabular}{c c c c c}
            \toprule
            RB mode & \multicolumn{2}{c}{Top qubit} & \multicolumn{2}{c}{Bottom qubit} \\
            \cmidrule(lr){2-3}
            \cmidrule(lr){4-5}
            & $T_1^2 (\mu s)$ & $\alpha$ (\%) & $T_1^2$ $(\mu s)$ & $\alpha$ (\%) \\
            \midrule
            Alternating  & $10.0(6)$    & $0.027(1)$ & $4.9(1.9)$  & $0.011(5)$\\
            Compiled     & $8.8(6)$    & $0.024(1)$  & $4.5(9)$ & $0.011(2)$\\
            5 primitives &$9.7(7)$    & $0.038(3)$  & $4.9(1)$  & $0.018(5)$\\
            \bottomrule
          \end{tabular}
        \end{table}


        \begin{figure}[h]
          \centering
          \includegraphics[width=\textwidth]{Figures/Randomized benchmarking/2nd-state leakage 2Q.png}
          \caption{Second-excited-state leakage along with fit using Eq.~\ref{eq:2nd state formula}.}
          \label{fig:second state leakage 2Q}
        \end{figure}

\chapter{Compiled randomized benchmarking algorithm}
  \label{sec:compiled randomized benchmarking algorithm}
  \section{Finding the optimal pulse sequence}
    In compiled randomized benchmarking, every combination of $n$ Cliffords, each applied to a separate qubit, is compiled to minimize the total number of pulses. The constraint is that distinct pulses may not be applied simultaneously. Any one pulse may, however, can be directed to any subset of the $n$ qubits. The compilation is done by comparing all possible pulse decompositions for each of the $n$ Cliffords with each other, and determining the sequence of pulses that results in the minimum number of total pulses.

    Our algorithm for finding the minimum number of pulses for a particular combination of Clifford decompositions is recursive. It determines all possible ways in which the pulses can be ordered, and chooses the one with the minimum number of pulses. Given $n$ Cliffords $\left(C_{\alpha_1}, \dots, C_{\alpha_n}\right)$, where $\alpha_i$ is the Clifford ID for qubit $i$, a particular combination of Clifford decompositions is given by $\left(\left( P_1^1, ..., P_{m_1}^1 \right) , ..., \left(P_1^n, ..., P_{m_n}^n\right)\right)$, where $m_i$ is the number of pulses in the decomposition of Clifford $C_{\alpha_i}$, and $P_j^i$ is pulse $j$ of the particular decomposition of $C_i$. To explain this algorithm, let us denote $\bm{\beta}=\left(\beta_1, \dots, \beta_n\right)$ as the indices of the next possible pulses in the decompositions, with corresponding pulses $\bm{P_\bm{\beta}}=\left( P_{\beta_1}^1, \dots, P_{\beta_n}^n \right)$. For the indices where $\beta_i=m_i + 1$, there is no next pulse, and so $P_{\beta_i}$ does not exist, and should therefore not be added to $\bm{P_\bm{\beta}}$. The algorithm operates as follows:

    \begin{enumerate}
      \item Start with an empty sequence of pulses $P_\text{seq}$ and pulse indices $\bm{\beta} = \left(\beta_1, ..., \beta_n\right) = \left(1, ..., 1\right)$;
      \item Determine the set of distinct pulses in $P_{\bm{\beta}}$;
      \item For each pulse $P$ in $P_{\bm{\beta}}$, perform the following steps:
      \begin{enumerate}
        \item Append $P$ to $P_\text{seq}$;
        \item Copy pulse indices $\bm{\beta}$ to $\bm{\beta}^\text{new}=\left(\beta_1^\text{new}, ..., \beta_n^\text{new}\right)$;
        \item For all indices $i$ for which $P_{\beta_i}^i=P$, increase the pulse index $\beta_i^\text{new} = \beta_i+1$;
        \item Go to step 2 using the new pulse indices $\bm{\beta} \rightarrow \bm{\beta}^\text{new}$;
      \end{enumerate}
      \item When $P_{\bm{\beta}}$ is empty, $P_\text{seq}$ is a particular sequence of pulses such that all Cliffords are applied to the corresponding qubits;
      \item After considering all possible pulse sequences, choose the sequence with the minimum number of pulses in $P_\text{seq}$.
    \end{enumerate}

    This algorithm determines the minimum number of pulses required to perform all Cliffords for one particular combination of Clifford decompositions. However, each Clifford has an average of $38$ different decompositions, and so for $n$ qubits there are approximately $38^n$ different combinations of Clifford decompositions. Furthermore, applying the algorithm to all $38^n$ decomposition combinations would only determine an optimal Clifford compilation for one particular Clifford combination; the average number of pulses per Clifford combination is found by averaging over all $24^n$ possible Clifford combinations. This problem scales exponentially with $n$, and for $n=5$ qubits, as many as $24^5\cdot38^5 \approx 6.3\cdot10^{14}$ combinations of Clifford decompositions need to be considered. Nevertheless we have exactly calculated the average number of pulses per Clifford combination for up to $n=5$ qubits. For this, we employed several optimizations, that are discussed below.

  \section{Optimizing the Clifford compilation algorithm}
    \label{Optimizing the Clifford compilation algorithm}

    The first and simplest optimization arises from the observation that an optimal Clifford compilation for a certain Clifford combination $\left(C_{\alpha_1}, \dots, C_{\alpha_n}\right)$ is the same as for any permutation of those Cliffords. We therefore only determine an optimal Clifford compilation when $\beta_1 \leq \dots \leq \beta_n$. This already reduces the number of calculations exponentially ($81$ times less computations when $n=5$).

    In the second optimization, we place an upper bound $N^\text{ub}$ on the pulse sequence length. the upper bound $N^\text{ub}$ is given by the minimum number of pulses found so far that can compile a given Clifford combination. At each stage, the algorithm checks if the sum of the pulses in $P_\text{seq}$ and all distinct pulses left is equal to or greater than $N^\text{ub}$. If this is the case, a shorter combination of pulses using $P_\text{seq}$ is not possible. It therefore stops considering this sequence and proceeds to the next one. Initially $N^\text{ub}=5$, as the $5$-primitives method proves that there is always a decomposition of an arbitrary number of Cliffords into $5$ pulses. Note that, as $N^\text{ub}$ decreases, the frequency at which the algorithm stops considering sequences increases.

    The third optimization relies on decompositions with fewer pulses being more likely to result in an optimal Clifford compilation. The decompositions of every Clifford are therefore arranged in ascending number of pulses. The first decompositions compared are then those with the minimum number of pulses; these have the highest probability of finding an optimal Clifford compilation. Even if an optimal Clifford compilation is not found, it is more likely that $N^\text{ub}$ will be low. This optimization is especially effective in combination with the second optimization.

    The fourth optimization places a lower bound $N^\text{lb}$ on the number of pulses. For a given Clifford combination $\left(C_{\alpha_1}, \dots, C_{\alpha_n}\right)$, $N^\text{lb}$ is found by looking at the minimum number of pulses $N^\text{opt}$ previously found for all $n-1$ Clifford subsets. Since $N^\text{opt}$ for the $n$ Cliffords can never be less than $N^\text{opt}$ for any of the $n-1$ Clifford subsets, the maximum length of the $n-1$ Clifford subsets therefore places a lower bound $N^\text{lb}$ on $N^\text{opt}$ for the $n$ Cliffords. This means that if a pulse sequence is found whose length is equal to this $N^\text{lb}$, it is an optimal Clifford compilation, and all further search is aborted. This is in contrast to the second optimization, where only the particular sequence of pulses is aborted upon reaching $N^\text{ub}$. Furthermore, as $n$ increases, it becomes increasingly likely that the lower bound is equal to 5. In this case $N^\text{lb}=N^\text{ub}$, and so the $5$-primitives method is an optimal Clifford compilation. This optimization results in a largest gain in computation time, by several orders of magnitude.

    In the fifth, and most complicated, optimization, all decompositions composed of three pulses or less are separated from those composed of four pulses. First, all combinations of Clifford decompositions composed of three pulses or less are compared. This reduces the average number of decompositions per Clifford, from $38$ to $7$, resulting in an exponentially reduced number of total decomposition combinations. It is, however, not always the case that the optimal Clifford compilation is found using only up to three pulses per decomposition; sometimes optimal Clifford compilations requires that one of the decompositions is composed of four pulses. However, after comparing decompositions of three pulses or less, these four-pulse decompositions only need to be considered when $N^\text{lb} \leq 4$ and $N^\text{ub} =5$. If there is a sequence containing a four-pulse decomposition that outperforms any found using up to three-pulse decompositions and the $5$-primitives method, the sequence must consist of 4 pulses. Only one Clifford then has a four-pulse decomposition, while all other Cliffords are subsets of these four pulses. We therefore loop, for every Clifford, over each of the four-pulse decompositions, and test whether every other Cliffords can be decomposed into a subset of these four pulses. This changes the comparison of four-pulse decompositions from scaling exponentially with $n$ to scaling linearly.

    \begin{table}
      \begin{tabular}{c c || c c}
        \toprule
        number of qubits  & average pulses per Clifford & number of qubits  & average pulses per Clifford \\
        \midrule
        1 & 1.875 & 6  & 4.380 (12)\\
        2 & 2.925 & 7  & 4.570 (15)\\
        3 & 3.521 & 8  & 4.721 (10)\\
        4 & 3.874 & 9  & 4.808 (14)\\
        5 & 4.137 & 10 & 4.857 (24)\\
        \bottomrule
        %random sampling 5: 4.139(2)
      \end{tabular}
      \caption{The average pulses per Clifford after compilation. Values up to $n=5$ have been calculated exact, while values starting from $n=6$ have been determined using random sampling.}
      \label{tab:pulses per Clifford}
    \end{table}

    Using these five optimizations, we have calculated the average pulses per Clifford combination exactly for $n=5$ qubits within two hours. Furthermore, the average pulses per Clifford has been approximated for up to $n=10$ Cliffords using random sampling. The results are shown in Table~\ref{tab:pulses per Clifford}.
