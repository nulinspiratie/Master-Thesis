% Todo:
% Change all the variables such that they are not in italics

\documentclass[12pt]{report}
\usepackage[a4paper]{geometry}
\usepackage[myheadings]{fullpage}
\usepackage{amsmath}
\usepackage{fancyhdr}
\usepackage{lastpage}
\usepackage{graphicx, wrapfig, subcaption, setspace, booktabs}
% \usepackage[T1]{fontenc}
\usepackage[font=small, labelfont=bf]{caption}
\usepackage{fourier}
\usepackage[protrusion=true, expansion=true]{microtype}
\usepackage[english]{babel}
\usepackage{sectsty}
\usepackage{url, lipsum}
\usepackage{siunitx}
\usepackage{subcaption}
\usepackage[space]{grffile}
\usepackage[toc, page]{appendix}
\usepackage{braket}

\newcommand{\Ec}{E_\text{c}}
\newcommand{\Ej}{E_\text{J}}
\newcommand{\fbare}{f_\text{bare}}
\newcommand{\wbare}{\omega_\text{bare}}
\newcommand{\fres}{f_\text{r}}
\newcommand{\wres}{\omega_\text{r}}
\newcommand{\fqub}{f_\text{q}}
\newcommand{\wqub}{\omega_\text{q}}

\newcommand{\HRule}[1]{\rule{\linewidth}{#1}}
% \onehalfspacing
\setcounter{secnumdepth}{5}
\setcounter{tocdepth}{5}

%-------------------------------------------------------------------------------
% HEADER & FOOTER
%-------------------------------------------------------------------------------
\pagestyle{fancy}
\fancyhf{}
\setlength\headheight{15pt}
\fancyhead[L]{Serwan Asaad}
\fancyhead[R]{Delft University of Technology}
\fancyfoot[R]{Page \thepage\ of \pageref{LastPage}}
%-------------------------------------------------------------------------------
% TITLE PAGE
%-------------------------------------------------------------------------------

\begin{document}

\begin{titlepage}
\begin{center}
~\\ [4.0 cm]
\textsc{\LARGE Delft University of Technology}
\\ [3.0 cm]
\textsc{\Large Master Thesis}
\HRule{0.5 pt} \\
\LARGE \textbf{\uppercase{Master thesis}}
\HRule{2 pt} \\ [0.5 cm]

% Author and supervisor
\noindent
\begin{minipage}{0.4\textwidth}
\begin{flushleft} \large
\emph{Author:}\\
Serwan Asaad
\end{flushleft}
\end{minipage}%
\begin{minipage}{0.4\textwidth}
\begin{flushright} \large
\emph{Supervisor:} \\
Dr.~Alessandro Bruno
\end{flushright}
\end{minipage}
\\ [3.0 cm]
{\large \today}
\end{center}

\end{titlepage}


\author{
    Serwan Asaad
    Student ID: 4323475 \\
    Delft University of Technology \\
    Kavli Institute of Nanoscience\\
    Quantum Nanoscience Department\\
    Quantum Transport Group\\
    DiCarlo Lab}

\tableofcontents
\newpage

%-------------------------------------------------------------------------------
% Section title formatting
\sectionfont{\scshape}
%-------------------------------------------------------------------------------

%-------------------------------------------------------------------------------
% BODY
%-------------------------------------------------------------------------------

\section*{Introduction}

\part{Deep-reactive ion etched resonators}

\textbf{TODO:}
\begin{itemize}
  \item Explain heterodyne detection
  \item Explain VNA
  \item Explain that resonance frequency $\fres$ is not necessarily at the transmission minimum when the resonator exhibits asymmetry.
\end{itemize}





\part{Muxmon experiment}

  \textbf{TODO:} Explain surface code architecture



  \chapter*{Introduction}

    At this moment circuit QED is at the stage where multi-qubit experiments are being realized.



  \chapter{Muxmon chip architecture}

    \textbf{Topics that should be explained in this section:}
    \begin{itemize}
      \item The Muxmon0 and Muxmon1 chip are designed with two purposes
      \begin{enumerate}
        \item Testing multiplexing using the Duplexer
        \item Explore qubit frequency re-use
      \end{enumerate}

      \item Explain similarities of chips
      \begin{itemize}
        \item Three qubits per chip
        \item All three qubits have individual flux tuning
        \item Air bridges are used, not only for connect the ground planes, but also such that the feed line can pass over other coplanar waveguides without contact \\
      \end{itemize}

      \item Explain differences between Muxmon0 and Muxmon1.
      \begin{itemize}
        \item The Muxmon0 chip has a driving line connected to each of the qubits. \\
            It has two resonator buses at \SI{4.9}{\giga \hertz} and \SI{5.0}{\giga \hertz}. \\
            These could also be used for two-qubit gates.
        \begin{description}
          \item[Advantage] Able to fully control each qubit individually, even when multiple qubits share the same frequency.
          \item[Advantage] Less coupling between data qubits.
          \item[Disadvantage] Requires more driving lines.
          \item[Disadvantage] Adds extra source of dissipation for the qubits.
        \end{description}
        \item The Muxmon1 chip has two driving lines, each capacitively coupled to one of the two data qubits, and to the ancilla qubit.
        \begin{description}
          \item[Advantage] Less driving lines required
          \item[Advantage] Less dissipation due to capacitive coupling
          \item[Disadvantage] Cannot individually control data qubit and ancilla qubit when they share the same frequency
          \item[Disadvantage] More coupling between qubits
        \end{description}
        \item Simplified model of the surface code
      \end{itemize}

      \item Explain concepts of cross-coupling and readout cross-talk
      \begin{description}
        \item[Cross-coupling] The coupling between qubits. \\
                    Cross-coupling leads to transfer of excitation.\\
                    An associated coupling strength \textbf{g} can be associated to cross-coupling.\\
                    Can be determined by driving one qubit extremely hard, and measuring signal from other qubit.\\
                    \textbf{TODO:} Show values of cross-coupling found, or do this in characterization section \\
                    \textbf{TODO:} Leads to coherent errors? \\
                    \textbf{TODO:} Two types of cross-coupling? Direct leakage of pulse pulse, and transfer of excitation? cross-driving?
        \item[Readout cross-talk] Coupling between a qubit and a resonator that are not directly coupled.\\
                      A part of the signal measured from one resonator is then due to the state of another qubit \\
                      \textbf{TODO:} Understand more behind readout cross-talk
      \end{description}

    \end{itemize}

    \textbf{Left to think about:}
    \begin{itemize}
      \item Should I already include items such as coherence times, the fact that Muxmon0 performs better than Muxmon 1?
      \item Where should I include coherence times versus frequency?
      \item Should the part on cross-coupling and readout cross-talk not be in characterization section?
      \item Should the section on the Duplexer go in here?
    \end{itemize}

    \textbf{Figures that need to be included:}
    \begin{itemize}
      \item Muxmon0 and Muxmon1 chip, preferably optical microscopy
      \item SEM image of air-bridges such that coplanar wave-guides cross without intersecting
      \item schematic of cross-coupling and readout cross-talk \\
          It could be good to create this using the actual Muxmon chip as background, with arrows indicating how the different effects operate
    \end{itemize}


  \chapter{Qubit characterization}
    \begin{description}
     \item[Description] This chapter gives a step-by-step description of how to find a resonator and qubit, and subsequently how to tune the qubit's parameters.
     \end{description}

    In the design of cQED chips, the parameters of the qubits and resonators are always targeted which are ideal for the experiment.
    For coplanar waveguide resonators one can already obtain relatively good parameters for the required dimensions from simple formulae \textbf{TODO:} refer to formula.
    For superconducting transmon qubits, however, finding the right dimensions that correspond to the desired parameters is a much more complicated process.
    The qubit's frequency, for instance, depends on the qubit's coupling energy $\Ec$ and Josephson energy $\Ej$. The coupling energy $\Ec$ can be reasonably estimated from classical simulations. Finding the right dimensions for the Josephson junction that result in the desired Josephson energy $\Ej$, however, is difficult, and usually physically testing different junctions is necessary to determine an accurate conversion from the desired $\Ej$ to the Josephson junction dimensions.

    Nevertheless, the actual parameters of the resonators and qubits are almost never where one expects them to be. Once the sample is cooled in the dilution refrigerator, an inevitable game of hide-and-seek follows with the goal of finding the frequency of the resonators and qubits, and subsequently determining their properties. This chapter describes the measurements that were performed to characterize the MuxMon samples.

    \section{Part I: Continuous-wave measurements}

      Once a sample is properly cooled down it is ready to be measured. At this stage the sample is still an unknown terrain, where the experimenter only has a rough map, containing the sample's targeted parameters, and the specific properties of the resonators and qubits.

      The first step is to look for signs of life. These manifest themselves as resonance frequencies of the resonators and the qubits that are coupled to them. As we are not yet interested in the properties of the resonators and qubits which can only be obtained through measurements with accurately timed pulses, we send continuous tones through the feedline, and measure deviations in the transmission. These measurements are known as continuous-wave measurements

      \subsection{Scanning for resonators}
        Since communication with the qubits is mediated through their coupling to resonators, the first step is to find these resonators. This is done using a transmission measurement, in combination with heterodyne detection, and has been explained in section \textbf{TODO:} Create section in Resonator chapter.

        There is one difference in measuring a resonator when there is a qubit coupled to it. When considering the qubit as a two-level system, the behaviour of the coupled resonator-qubit system is governed by the Jaynes-Cummings Hamiltonian \cite{Reed}:

        \begin{equation}
          \hat{H} = \hbar \wres\left(\hat{a}^\dagger \hat{a} + \frac{1}{2} \right) + \frac{\hbar \wqub}{2}\hat{\sigma}_z + \hbar g \left(\hat{a}^\dagger \sigma_{-} + \hat{a}\sigma_{+}\right)
          \label{eq:Jaynes-Cummings}
        \end{equation}

        where $\wres$ is the bare resonance frequency of the resonator, $\wqub$ is the resonance frequency of the qubit's ground to excited state transition, and the qubit's two states are in the spin-representation. This Hamiltonian consists of three terms. The first term corresponds to the energy level of the resonator, the second to the energy level of the transmon, and the third is a coupling term between the two with coupling strength $g$.

        The difference between the resonator's frequency $\wres$ and the qubit's frequency $\wqub$ is given by the detuning $\Delta = \wqub - \wres$. If the magnitude of the detuning is large compared to the coupling strength $g$, the system is in the dispersive regime. In this case the Hamiltonian can be approximated by the dispersive Jaynes-Cummings Hamiltonian:

        \begin{equation}
          \hat{H} = \frac{\hbar \wqub^{'}}{2} \hat{\sigma}_z +  \left(\hbar \wres^{'} + \hbar \chi \hat{\sigma}_z\right) \hat{a}^\dagger \hat{a}
          \label{eq:dispersive-Jaynes-Cummings}
        \end{equation}

        The coupling between the qubit and resonator causes both qubit's frequency and the resonator's frequency to shift: $\wqub^{'} = \wqub + \chi_{01}$, $\wres^{'}=\wres - \chi_{12}/2$.

        Aside from experiencing a frequency shift dependent on the amount of detuning, Equation~\ref{eq:dispersive-Jaynes-Cummings} shows that the resonator also experiences a shift depending on the state of the qubit. The resonator's frequency is decreased by an amount $2 \chi$ when the qubit is in the excited state. The parameter $\chi$ is the dispersive shift, and is given by:

        \begin{equation}
          \chi = \chi_{01} - \chi_{12}/2 \approx \frac{g^2}{\Delta}\frac{\Ec}{\hbar \Delta - \Ec}
          \label{eq:dispersive-shift}
        \end{equation}

        where $\chi_{ij} = \frac{g_{ij}^2}{\omega_{ij}-\omega_c}$ are the partial dispersive shifts.

        Due to this coupling between resonator and qubit, it is important to choose the right RF power. When the amount of photons in the resonator reaches a certain point, this coupling will result in the resonator experiencing nonlinear effects. The resonator will thereby lose its Lorentzian lineshape. Therefore the RF power should be kept sufficiently low to avoid these nonlinear effects, while still maintaining a good signal-to-noise ratio.



        \textbf{TODO:}
        \begin{itemize}
          \item In the strong coupling regime (Leads to hybridization of qubit and resonator states: quton and fobit
          \item Explain that the quality factor is low because the resonator is coupled strongly to the qubit (equation including coupling to qubit?)
          \item Doesn't $\chi$ diverge when $\Delta \rightarrow \Ec$?
          \item explain concept of anharmonicity, and that $\alpha \approx E_c$
          \item Maybe explain concept of number splitting in dispersive Jaynes-Cummings. Number splitting is the phenomenon that the qubit's frequency shifts by an amount $2 \chi$ for every photon in the resonator.\\
                Alternatively mention this in another section.
          \item Mention that $g_{12}=\sqrt{g}$, and that other coupling strengths are exponentially suppressed in the transmon
        \end{itemize}

        \textbf{Figures:}
        \begin{itemize}
          \item Figure of transmission showing all three Muxmon0 resonators
        \end{itemize}

      \subsection{Powersweeping the resonators}
        Once the resonators have been located, the next stage is to find the qubit that is capacitively coupled to each of the resonators. Instead of directly scanning the entire frequency spectrum in search of the qubit, it is relatively straightforward to perform some initial measurements aimed at gaining information about our resonator and qubit, which will allow us to search for our qubit with much greater accuracy.

        As explained previously, the capacitive coupling between the resonator and qubit shifts the resonator frequency $\wres$ from its bare frequency. When the amount of photons in the resonator reaches a certain point, the resonator experiences nonlinearity, thereby losing its Lorentzian lineshape. When increasing the RF power even further, at a certain point the resonator regains its Lorentzian lineshape. In doing so its resonance frequency has shifted to its bare frequency $\wbare$. If this frequency shift is observed, it indicates that the resonator's frequency was shifted, and hence that the qubit is alive. Measuring this frequency shift is commonly done in a powersweep. A powersweep is a measurement in which a resonator scan is performed for a range of powers.

        A powersweep additionally provides information about at what power the resonator enters the nonlinear regime. For measurements involving the qubit the readout power must be below this threshold power. Furthermore, from the frequency shift between the dressed cavity frequency and the bare cavity frequency, the amount of detuning between the qubit and the resonator can be estimated using Equation~\ref{eq:dispersive-shift}.

        If no shift is observed, it could mean that the qubit is dead (e.g. because the Josephson junction is shorted). However, this is not necessarily the case. An alternative possibility is that the detuning between qubit and resonator is very large, and as a result the frequency shift cannot be discerned. At this point it is too early to draw conclusions, and we may almost draw the analogy with Schr\"odinger's cat in a box.

        \textbf{TODO:}
        \begin{itemize}
          \item Explain theory behind transition to bare cavity frequency (Reed's thesis has some information)
        \end{itemize}

        \textbf{Figures:}
        \begin{itemize}
          \item Powersweep of ancilla qubit
        \end{itemize}

      \subsection{Scan for qubit sweet-spots}
        Some of the qubits have a tunable resonance frequency. This is done through a superconducting quantum interference device (SQUID). In this case the two islands that compose the transmon qubit are connected by two Josephson junctions instead of one, effectively forming a loop. The SQUID loop is sensitive to the amount of flux passing through the loop. The amount of flux going through the SQUID loop can be changed by changing the surrounding magnetic field. This is commonly done by having a flux bias line in close proximity to the SQUID loop. Current flowing through the flux-bias line alters the magnetic field in the vicinity of the SQUID loop, and hence changes the amount of flux through the SQUID loop. A digital-to-analog converter is used to specify the amount of current that is sent through the flux bias line. Depending on the amount of flux through the SQUID loop, the resonance frequency of the qubit changes accordingly. these qubits are therefore called flux-tunable.

        For qubits that are flux-tunable, finding the sweet-spot of the qubit can be done without knowledge of the qubit's frequency. This can be done by sweeping the DAC voltage and measuring the shift in the resonance frequency. Because the frequency of the qubit varies as the amount of current through the flux-bias line changes, the detuning between the qubit and the resonator consequently changes. As a result the dispersive shift $\chi$, and therefore the resonator's frequency, also varies. At the sweet-spot of the qubit, the resonator's frequency $\wres$ is at a maximum. This is irrespective of whether the qubit's frequency $\wqub$ is above or below the resonator's frequency.

        The accurate way to measure the sweet-spot is to perform resonator scans as the DAC voltage is varied. The result is a 2D scan shown in \textbf{TODO:} Figure.

        A faster second approach for finding the qubit sweet-spot, at the cost of providing less information, is by choosing a fixed frequency close to the resonator's frequency $\wres$ (preferrably slightly below, where the transmission slope is steepest). By measuring the amount of transmission as the DAC voltage is being varied, one obtains essentially a line-cut of \textbf{TODO:} Figure. The idea this measurement is that if the qubit's frequency $\fqub$ decreases, the resonator's frequency also decreases, resulting in a decrease in transmission (closer to $\wres$). Likewise, if the qubit's frequency $\wqub$ increases, the resonator's frequency increases \textbf{TODO:} Why?, resulting in an increase in transmission (further away from $\wres$).

        At the qubit's sweet-spot, the resonator's frequency $\wres$ is at a maximum, and so the transmission should also be at a maximum. Furthermore, because the amount of detuning only depends on the deviation from the flux sweet-spot \textbf{TODO:} improve, the transmission should be symmetric with respect to the DAC voltage sweet-spot. If the resonator's frequency $\wres$ shifts by a large amount in the course of this measurement, it becomes harder to determine where the sweet-spot is (although even then often it can still be discerned). Nevertheless, this method is considerably faster than performing a full two-dimensional scan of frequency versus DAC voltage, and in most cases it works like a charm.

        In the case where the powersweep showed no measurable frequency shift, these two measurements are also useful in discerning whether or not the qubit is actually dead, or whether it was simply far detuned from the resonator.


        \textbf{TODO:}
        \begin{itemize}
          \item Explain how qubit's frequency has a cosine dependence on DAC
          \item Give detailed information on SQUID loop
          \begin{itemize}
            \item Why does the qubit frequency change in a SQUID loop
            \item Sweet-spot
          \end{itemize}
          \item Mention flux-noise?
          \begin{itemize}
            \item 1/f noise
            \item usually not limiting, as it is very slow
            \item This noise can be seen as occasional jumps (every few hours?) It would mean that every few hours the frequency must be recalibrated.
          \end{itemize}
        \end{itemize}

        \textbf{Figures:}
        \begin{itemize}
          \item SEM picture of SQUID loop, including flux-bias line
          \item Figure of 2D resonator scan vs DAC voltage
          \item Figure of 1D resonator fixed frequency DAC voltage scan
        \end{itemize}

      \subsection{Scanning for qubits}
        Once the preliminary measurements have been performed that characterize the resonators and provide hints about the whereabouts of the qubit it is time to actually find its frequency.

        The measurement to perform in order to find the qubit depends on the amount of detuning between the resonator and qubit, which can be estimated from powersweep measurements. If the amount of detuning is small ($ < \sim \SI{100}{\mega \hertz}$), the frequencies of the qubit and resonator are close to the avoided crossing, resulting in a strong coupling between the two. In this case the qubit can be seen in a simple transmission measurement. If the detuning between the resonator and qubit is sufficiently large ($> \sim \SI{100}{\mega \hertz}$), the qubit can be found using two-tone spectroscopy

        \subsubsection{Avoided crossing}
          When the qubit's frequency is close to that of the resonator, the two experience a strong coupling \textbf{TODO:} Why?

          \textbf{Avoided crossing info:}
          \begin{itemize}
            \item Actually the threshold is determined by the ratio between detuning and coupling strength
            \item The coupling strength $g$ is the minimum distance be~tween the splitting
            \item From Reed's thesis p.63 explanation of this avoided crossing is given\\
            \begin{align}
             E_0 = & -\frac{\hbar \Delta}{2}\\
             E_1 = & n \hbar\omega_r \pm \frac{\hbar}{2}\sqrt{4g^2n + \Delta^2}
            \end{align}\\
            Joining $E_0 + E_1$ results in a qubit approaching a resonator from the top.\\
            When the frequency of the qubit equals that of the resonator, the energy difference reaches a minimum, and is equal to $2g$.
          \end{itemize}
          \textbf{TODO:}
          \begin{itemize}
            \item Explain the quton fobit behaviour near the avoided crossing
            \item Explain how one can extract coupling from avoided crossing
          \end{itemize}

          \textbf{figures:}
          \begin{itemize}
            \item Figure of avoided crossing
          \end{itemize}

        \subsubsection{Spectroscopy}
          An increase in the detuning between the resonator and qubit is accompanied by a . Therefore, when driving at the qubit's frequency, the deviation in the transmission becomes less, and at a certain amount of detuning it is not measurable anymore.

          Explain in this Spectroscopy section:
          \begin{itemize}
              \item
          \end{itemize}


        If the qubit is flux-tunable, one important question to ask is: at what DAC voltage should the qubit be searched?

        There are several options. The first option is to choose zero DAC voltage. In most cases this is alright, but there are cases when this is a bad choice. For instance, due to trapped magnetic fields during cool-down the qubit could be positioned at the anti-sweet-spot for zero DAC voltage, in which case finding the qubit will be next to impossible.

        A better option is to choose a DAC voltage depending on the amount of detuning previously measured. Ideally either a few hundred megahertz or zero.

        \textbf{TODO:}
        \begin{itemize}
          \item Use relatively high source power, results in power broadening
          \item Mention what to do if qubit is close to or far away from resonator
          \item Mention the term transmission measurement earlier on.
        \end{itemize}




    \section{Part II: Time-domain measurements}








  \chapter{Calibration routines}


  \chapter{Randomized benchmarking}

\begin{appendices}

  \chapter{Additional notes}
    \section{Qubit characterization}
      \subsection{Part I: Continuous-wave measurements}
        \subsubsection{Powersweep}
          \begin{itemize}
            \item If the resonator seems to disappear after the transition from dressed frequency to bare frequency (or vice versa), this is likely due to the qubit being extremely close to the resonator, and so the frequency shift is large.
          \end{itemize}
        \subsubsection{Spectroscopy}
          \begin{itemize}
            \item If the deviation in transmission becomes less due to more detuning, increasing the power can also increase the contrast.
          \end{itemize}
  \chapter{Algorithms}
    \section{Tracked spectroscopy}
    \section{Peak finding}
    \section{Compiled RB?}

\end{appendices}

\bibliographystyle{plain}
\bibliography{bibliography}


\end{document}