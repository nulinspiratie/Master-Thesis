% Todo:
% Change all the variables such that they are not in italics

\documentclass[12pt]{report}
\usepackage[a4paper]{geometry}
\usepackage[myheadings]{fullpage}
\usepackage{amsmath}
\usepackage{fancyhdr}
\usepackage{lastpage}
\usepackage{graphicx, wrapfig, subcaption, setspace, booktabs}
% \usepackage[T1]{fontenc}
\usepackage[font=small, labelfont=bf]{caption}
\usepackage{fourier}
\usepackage[protrusion=true, expansion=true]{microtype}
\usepackage[english]{babel}
\usepackage{sectsty}
\usepackage{url, lipsum}
\usepackage{siunitx}
\usepackage{subcaption}
\usepackage[space]{grffile}

\newcommand{\Ec}{$E_\text{c}$}
\newcommand{\Ej}{$E_\text{J}$}

\newcommand{\HRule}[1]{\rule{\linewidth}{#1}}
% \onehalfspacing
\setcounter{secnumdepth}{5}
\setcounter{tocdepth}{5}

%-------------------------------------------------------------------------------
% HEADER & FOOTER
%-------------------------------------------------------------------------------
\pagestyle{fancy}
\fancyhf{}
\setlength\headheight{15pt}
\fancyhead[L]{Serwan Asaad}
\fancyhead[R]{Delft University of Technology}
\fancyfoot[R]{Page \thepage\ of \pageref{LastPage}}
%-------------------------------------------------------------------------------
% TITLE PAGE
%-------------------------------------------------------------------------------

\begin{document}

\begin{titlepage}
\begin{center}
~\\ [4.0 cm]
\textsc{\LARGE Delft University of Technology}
\\ [3.0 cm]
\textsc{\Large Master Thesis}
\HRule{0.5 pt} \\
\LARGE \textbf{\uppercase{Master thesis}}
\HRule{2 pt} \\ [0.5 cm]

% Author and supervisor
\noindent
\begin{minipage}{0.4\textwidth}
\begin{flushleft} \large
\emph{Author:}\\
Serwan Asaad
\end{flushleft}
\end{minipage}%
\begin{minipage}{0.4\textwidth}
\begin{flushright} \large
\emph{Supervisor:} \\
Dr.~Alessandro Bruno
\end{flushright}
\end{minipage}
\\ [3.0 cm]
{\large \today}
\end{center}

\end{titlepage}


\author{
    Serwan Asaad
    Student ID: 4323475 \\
    Delft University of Technology \\
    Kavli Institute of Nanoscience\\
    Quantum Nanoscience Department\\
    Quantum Transport Group\\
    DiCarlo Lab}

\tableofcontents
\newpage

%-------------------------------------------------------------------------------
% Section title formatting
\sectionfont{\scshape}
%-------------------------------------------------------------------------------

%-------------------------------------------------------------------------------
% BODY
%-------------------------------------------------------------------------------

\section*{Introduction}

\part{Deep-reactive ion etched resonators}







\part{Muxmon experiment}

  \textbf{TODO:} Explain surface code architecture



  \chapter*{Introduction}

    At this moment circuit QED is at the stage where multi-qubit experiments are being realized.



  \chapter{Muxmon chip architecture}

    \textbf{Topics that should be explained in this section:}
    \begin{itemize}
      \item The Muxmon0 and Muxmon1 chip are designed with two purposes
      \begin{enumerate}
        \item Testing multiplexing using the Duplexer
        \item Explore qubit frequency re-use
      \end{enumerate}

      \item Explain similarities of chips
      \begin{itemize}
        \item Three qubits per chip
        \item All three qubits have individual flux tuning
        \item Air bridges are used, not only for connect the ground planes, but also such that the feed line can pass over other coplanar waveguides without contact \\
      \end{itemize}

      \item Explain differences between Muxmon0 and Muxmon1.
      \begin{itemize}
        \item The Muxmon0 chip has a driving line connected to each of the qubits. \\
            It has two resonator buses at \SI{4.9}{\giga \hertz} and \SI{5.0}{\giga \hertz}. \\
            These could also be used for two-qubit gates.
        \begin{description}
          \item[Advantage] Able to fully control each qubit individually, even when multiple qubits share the same frequency.
          \item[Advantage] Less coupling between data qubits.
          \item[Disadvantage] Requires more driving lines.
          \item[Disadvantage] Adds extra source of dissipation for the qubits.
        \end{description}
        \item The Muxmon1 chip has two driving lines, each capacitively coupled to one of the two data qubits, and to the ancilla qubit.
        \begin{description}
          \item[Advantage] Less driving lines required
          \item[Advantage] Less dissipation due to capacitive coupling
          \item[Disadvantage] Cannot individually control data qubit and ancilla qubit when they share the same frequency
          \item[Disadvantage] More coupling between qubits
        \end{description}
        \item Simplified model of the surface code
      \end{itemize}

      \item Explain concepts of cross-coupling and readout cross-talk
      \begin{description}
        \item[Cross-coupling] The coupling between qubits. \\
                    Cross-coupling leads to transfer of excitation.\\
                    An associated coupling strength \textbf{g} can be associated to cross-coupling.\\
                    Can be determined by driving one qubit extremely hard, and measuring signal from other qubit.\\
                    \textbf{TODO:} Show values of cross-coupling found, or do this in characterization section \\
                    \textbf{TODO:} Leads to coherent errors? \\
                    \textbf{TODO:} Two types of cross-coupling? Direct leakage of pulse pulse, and transfer of excitation? cross-driving?
        \item[Readout cross-talk] Coupling between a qubit and a resonator that are not directly coupled.\\
                      A part of the signal measured from one resonator is then due to the state of another qubit \\
                      \textbf{TODO:} Understand more behind readout cross-talk
      \end{description}

    \end{itemize}

    \textbf{Left to think about:}
    \begin{itemize}
      \item Should I already include items such as coherence times, the fact that Muxmon0 performs better than Muxmon 1?
      \item Where should I include coherence times versus frequency?
      \item Should the part on cross-coupling and readout cross-talk not be in characterization section?
      \item Should the section on the Duplexer go in here?
    \end{itemize}

    \textbf{Figures that need to be included:}
    \begin{itemize}
      \item Muxmon0 and Muxmon1 chip, preferably optical microscopy
      \item SEM image of air-bridges such that coplanar wave-guides cross without intersecting
      \item schematic of cross-coupling and readout cross-talk \\
          It could be good to create this using the actual Muxmon chip as background, with arrows indicating how the different effects operate
    \end{itemize}


  \chapter{Qubit characterization}
    This chapter gives a step-by-step description of how to find a resonator and qubit, and subsequently how to tune the qubit's parameters.

    In the design of cQED chips, the parameters of the qubits and resonators are always targeted which are ideal for the experiment.
    For coplanar waveguide resonators one can already obtain relatively good parameters for the required dimensions from simple formulae \textbf{TODO:} refer to formula.
    For superconducting transmon qubits, however, finding the right dimensions that correspond to the desired parameters is a much more complicated process.
    The qubit's frequency, for instance, depends on the qubit's coupling energy \Ec and Josephson energy \Ej. The coupling energy \Ec can be reasonably estimated from classical simulations. Finding the right dimensions for the Josephson junction that result in the desired Josephson energy \Ej, however, is difficult, and usually physically testing different junctions is necessary to determine an accurate conversion from the desired \Ej to the Josephson junction dimensions.

    Nevertheless, the actual parameters of the sample are almost never where one expects them to be. Once the sample is cooled in the dilution refrigerator, an inevitable hide-and-seek game follows with the goal of finding the frequency of the resonators and qubits, and subsequently determining their properties. This section aims to describe the necessary operations to follow once a cQED sample is cooled down and ready to be characterized.

    \section{Part I: Continuous wave measurements}

     Once a sample is properly cooled down it is ready to be measured. At this stage the sample is still unknown terrain, where the experimenter only has a crude map, containing the sample's targeted parameters, and the specific properties of the resonators and qubits.

     Since the qubits









  \chapter{Calibration routines}


  \chapter{Randomized benchmarking}

\end{document}