% Todo:
% Change all the variables such that they are not in italics

\documentclass[12pt]{report}
\usepackage[a4paper]{geometry}
\usepackage[myheadings]{fullpage}
\usepackage{amsmath}
\usepackage{fancyhdr}
\usepackage{lastpage}
\usepackage{graphicx, wrapfig, subcaption, setspace, booktabs}
% \usepackage[T1]{fontenc}
\usepackage[font=small, labelfont=bf]{caption}
\usepackage{fourier}
\usepackage[protrusion=true, expansion=true]{microtype}
\usepackage[english]{babel}
\usepackage{sectsty}
\usepackage{url, lipsum}
\usepackage{siunitx}
\usepackage{subcaption}
\usepackage[space]{grffile}
\usepackage[toc, page]{appendix}

\newcommand{\Ec}{$E_\text{c}$}
\newcommand{\Ej}{$E_\text{J}$}
\newcommand{\fbare}{$f_\text{bare}$}
\newcommand{\fres}{$f_\text{r}$}
\newcommand{\fqub}{$f_\text{q}$}

\newcommand{\HRule}[1]{\rule{\linewidth}{#1}}
% \onehalfspacing
\setcounter{secnumdepth}{5}
\setcounter{tocdepth}{5}

%-------------------------------------------------------------------------------
% HEADER & FOOTER
%-------------------------------------------------------------------------------
\pagestyle{fancy}
\fancyhf{}
\setlength\headheight{15pt}
\fancyhead[L]{Serwan Asaad}
\fancyhead[R]{Delft University of Technology}
\fancyfoot[R]{Page \thepage\ of \pageref{LastPage}}
%-------------------------------------------------------------------------------
% TITLE PAGE
%-------------------------------------------------------------------------------

\begin{document}

\begin{titlepage}
\begin{center}
~\\ [4.0 cm]
\textsc{\LARGE Delft University of Technology}
\\ [3.0 cm]
\textsc{\Large Master Thesis}
\HRule{0.5 pt} \\
\LARGE \textbf{\uppercase{Master thesis}}
\HRule{2 pt} \\ [0.5 cm]

% Author and supervisor
\noindent
\begin{minipage}{0.4\textwidth}
\begin{flushleft} \large
\emph{Author:}\\
Serwan Asaad
\end{flushleft}
\end{minipage}%
\begin{minipage}{0.4\textwidth}
\begin{flushright} \large
\emph{Supervisor:} \\
Dr.~Alessandro Bruno
\end{flushright}
\end{minipage}
\\ [3.0 cm]
{\large \today}
\end{center}

\end{titlepage}


\author{
    Serwan Asaad
    Student ID: 4323475 \\
    Delft University of Technology \\
    Kavli Institute of Nanoscience\\
    Quantum Nanoscience Department\\
    Quantum Transport Group\\
    DiCarlo Lab}

\tableofcontents
\newpage

%-------------------------------------------------------------------------------
% Section title formatting
\sectionfont{\scshape}
%-------------------------------------------------------------------------------

%-------------------------------------------------------------------------------
% BODY
%-------------------------------------------------------------------------------

\section*{Introduction}

\part{Deep-reactive ion etched resonators}

\textbf{TODO:}
\begin{itemize}
  \item Explain heterodyne detection
  \item Explain VNA
  \item Explain that resonance frequency \fres is not necessarily at the transmission minimum when the resonator exhibits asymmetry.
\end{itemize}





\part{Muxmon experiment}

  \textbf{TODO:} Explain surface code architecture



  \chapter*{Introduction}

    At this moment circuit QED is at the stage where multi-qubit experiments are being realized.



  \chapter{Muxmon chip architecture}

    \textbf{Topics that should be explained in this section:}
    \begin{itemize}
      \item The Muxmon0 and Muxmon1 chip are designed with two purposes
      \begin{enumerate}
        \item Testing multiplexing using the Duplexer
        \item Explore qubit frequency re-use
      \end{enumerate}

      \item Explain similarities of chips
      \begin{itemize}
        \item Three qubits per chip
        \item All three qubits have individual flux tuning
        \item Air bridges are used, not only for connect the ground planes, but also such that the feed line can pass over other coplanar waveguides without contact \\
      \end{itemize}

      \item Explain differences between Muxmon0 and Muxmon1.
      \begin{itemize}
        \item The Muxmon0 chip has a driving line connected to each of the qubits. \\
            It has two resonator buses at \SI{4.9}{\giga \hertz} and \SI{5.0}{\giga \hertz}. \\
            These could also be used for two-qubit gates.
        \begin{description}
          \item[Advantage] Able to fully control each qubit individually, even when multiple qubits share the same frequency.
          \item[Advantage] Less coupling between data qubits.
          \item[Disadvantage] Requires more driving lines.
          \item[Disadvantage] Adds extra source of dissipation for the qubits.
        \end{description}
        \item The Muxmon1 chip has two driving lines, each capacitively coupled to one of the two data qubits, and to the ancilla qubit.
        \begin{description}
          \item[Advantage] Less driving lines required
          \item[Advantage] Less dissipation due to capacitive coupling
          \item[Disadvantage] Cannot individually control data qubit and ancilla qubit when they share the same frequency
          \item[Disadvantage] More coupling between qubits
        \end{description}
        \item Simplified model of the surface code
      \end{itemize}

      \item Explain concepts of cross-coupling and readout cross-talk
      \begin{description}
        \item[Cross-coupling] The coupling between qubits. \\
                    Cross-coupling leads to transfer of excitation.\\
                    An associated coupling strength \textbf{g} can be associated to cross-coupling.\\
                    Can be determined by driving one qubit extremely hard, and measuring signal from other qubit.\\
                    \textbf{TODO:} Show values of cross-coupling found, or do this in characterization section \\
                    \textbf{TODO:} Leads to coherent errors? \\
                    \textbf{TODO:} Two types of cross-coupling? Direct leakage of pulse pulse, and transfer of excitation? cross-driving?
        \item[Readout cross-talk] Coupling between a qubit and a resonator that are not directly coupled.\\
                      A part of the signal measured from one resonator is then due to the state of another qubit \\
                      \textbf{TODO:} Understand more behind readout cross-talk
      \end{description}

    \end{itemize}

    \textbf{Left to think about:}
    \begin{itemize}
      \item Should I already include items such as coherence times, the fact that Muxmon0 performs better than Muxmon 1?
      \item Where should I include coherence times versus frequency?
      \item Should the part on cross-coupling and readout cross-talk not be in characterization section?
      \item Should the section on the Duplexer go in here?
    \end{itemize}

    \textbf{Figures that need to be included:}
    \begin{itemize}
      \item Muxmon0 and Muxmon1 chip, preferably optical microscopy
      \item SEM image of air-bridges such that coplanar wave-guides cross without intersecting
      \item schematic of cross-coupling and readout cross-talk \\
          It could be good to create this using the actual Muxmon chip as background, with arrows indicating how the different effects operate
    \end{itemize}


  \chapter{Qubit characterization}
    \begin{description}
     \item This chapter gives a step-by-step description of how to find a resonator and qubit, and subsequently how to tune the qubit's parameters.
     \end{description}

    In the design of cQED chips, the parameters of the qubits and resonators are always targeted which are ideal for the experiment.
    For coplanar waveguide resonators one can already obtain relatively good parameters for the required dimensions from simple formulae \textbf{TODO:} refer to formula.
    For superconducting transmon qubits, however, finding the right dimensions that correspond to the desired parameters is a much more complicated process.
    The qubit's frequency, for instance, depends on the qubit's coupling energy \Ec and Josephson energy \Ej. The coupling energy \Ec can be reasonably estimated from classical simulations. Finding the right dimensions for the Josephson junction that result in the desired Josephson energy \Ej, however, is difficult, and usually physically testing different junctions is necessary to determine an accurate conversion from the desired \Ej to the Josephson junction dimensions.

    Nevertheless, the actual parameters of the sample are almost never where one expects them to be. Once the sample is cooled in the dilution refrigerator, an inevitable hide-and-seek game follows with the goal of finding the frequency of the resonators and qubits, and subsequently determining their properties. This chapter describes the measurements that were performed to characterize the MuxMon samples.

    \section{Part I: Continuous-wave measurements}

      Once a sample is properly cooled down it is ready to be measured. At this stage the sample is still an unknown terrain, where the experimenter only has a rough map, containing the sample's targeted parameters, and the specific properties of the resonators and qubits.

      The first step is to look for signs of life. These manifest themselves as resonance frequencies of the resonators and the qubits that are coupled to them. As we are not yet interested in the properties of the resonators and qubits which can only be obtained through measurements with accurately timed pulses, we send continuous tones through the feedline, and measure deviations in the transmission. These measurements are known as continuous-wave measurements

      \subsection{Scanning for resonators}
        Since communication with the qubits is mediated through their coupling to resonators, the first step is to find these resonators. This is done using heterodyne detection, and has been explained in section \textbf{TODO:} Create section in Resonator chapter.

        \textbf{TODO:}
        \begin{itemize}
          \item Explain why using a high power is good.
          \item Explain that the quality factor is low because the resonator is coupled strongly to the qubit (equation including coupling to qubit?)
        \end{itemize}

        \textbf{Figures:}
        \begin{itemize}
          \item Figure of transmission showing all three Muxmon0 resonators
        \end{itemize}

      \subsection{Powersweeping the resonators}
        Once the resonators have been located, the next stage is to find the qubit that is capacitively coupled to each of the resonators. Due to this capacitive coupling the frequency of the resonator experiences a shift dependent on the state of the qubit. This shift disappears when the resonator is driven with sufficient power, in which case its resonance frequency shifts to the bare cavity frequency \fbare. If this frequency shift is observed, it indicates that the qubit is alive. Measuring this frequency shift is commonly done in a powersweep. A powersweep is a measurement in which a resonator scan is performed for a range of powers.


        A powersweep additionally provides information about at what power the resonator enters the \textbf{TODO:} nonlinear? regime. For measurements involving the qubit the readout power must be below this threshold power. Furthermore, from the frequency shift between the dressed cavity frequency and the bare cavity frequency, it is possible to estimate the amount of detuning between the qubit and the resonator. This is done through the relation: \textbf{TODO:} insert relation

        If no shift is observed, it could mean that the qubit is dead (e.g. because the Josephson junction is shorted). However, this is not necessarily the case. An alternative possibility is that the resonance frequency of the qubit is sufficiently far away from that of the resonator, and as a result the frequency shift cannot be discerned. At this point it is too early to draw conclusions, and we may almost draw the analogy with Schr\"odinger's cat in a box.

        \textbf{TODO:}
        \begin{itemize}
          \item Explain better theory behind resonator shift $\chi$
          \item cosine formula for qubit curve.
          \item Explain theory behind transition to bare cavity frequency
          \item Refer to thesis Reed for more info on frequency shifts
        \end{itemize}

        \textbf{Figures:}
        \begin{itemize}
          \item Powersweep of ancilla qubit
        \end{itemize}

      \subsection{Scan for sweet-spots}
        Some of the qubits have a tunable resonance frequency. This is done through a superconducting quantum interference device (SQUID). In this case the two islands that compose the transmon qubit are connected by two Josephson junctions instead of one, effectively forming a loop. The SQUID loop is sensitive to the amount of flux passing through the loop. The amount of flux going through the SQUID loop can be changed by changing the surrounding magnetic field. This is commonly done by having a flux-bias line in close proximity to the SQUID loop. Current flowing through the flux-bias line alters the magnetic field in the vicinity of the SQUID loop, and hence changes the amount of flux through the SQUID loop. A digital-to-analog converter is used to specify the amount of current that is sent through the flux-bias-line. Depending on the amount of flux through the SQUID loop, the resonance frequency of the qubit changes accordingly. these qubits are therefore called flux-tunable.

        For qubits that are flux-tunable, finding the sweet-spot of the qubit can be done without knowledge of the qubit's frequency. This can be done by sweeping the DAC voltage and measuring the shift in the resonance frequency. Because the frequency of the qubit varies as the amount of current through the flux-bias line changes, the detuning between the qubit and the resonator consequently changes. As a result the resonator shift \textbf{TODO:} $\chi$? also varies. At the sweet-spot of the qubit, the resonance frequency of the resonator, which henceforth shall be denoted as the resonator's frequency \fres, is at a maximum. This is irrespective of whether the resonance frequency of the qubit's ground to excited state transition, which henceforth shall be denoted as the qubit's frequency \fqub, is above or below that of the resonator.

        The accurate way to measure the sweet-spot is to perform resonator scans as the DAC voltage is varied. The result is a 2D scan shown in \textbf{TODO:} Figure.

        A faster second approach for finding the qubit sweet-spot, at the cost of providing less information, is by choosing a fixed frequency close to the resonator's frequency \fres (preferrably slightly below, where the transmission slope is steepest). By measuring the amount of transmission as the DAC voltage is being varied, one obtains essentially a line-cut of \textbf{TODO:} Figure. The idea this measurement is that if the qubit's frequency \fqub decreases, the resonator's frequency also decreases, resulting in a decrease in transmission (closer to \fres). Likewise, if the qubit's frequency \fqub increases, the resonator's frequency increases \textbf{TODO:} Why?, resulting in an increase in transmission (further away from \fres).

        At the qubit's sweet-spot, the resonator's frequency \fres is at a maximum, and so the transmission should also be at a maximum. Furthermore, because the amount of detuning only depends on the deviation from the flux sweet-spot \textbf{TODO:} improve, the transmission should be symmetric with respect to the DAC voltage sweet-spot. If the resonator's frequency \fres shifts by a large amount in the course of this measurement, it becomes harder to determine where the sweet-spot is (although even then often it can still be discerned). Nevertheless, this method is considerably faster than performing a full two-dimensional scan of frequency versus DAC voltage, and in most cases it works like a charm.

        In the case where the powersweep showed no measurable frequency shift, these two measurements are also useful in discerning whether or not the qubit is actually dead, or whether it was simply far detuned from the resonator.


        \textbf{TODO:}
        \begin{itemize}
          \item Give detailed information on SQUID loop
          \begin{itemize}
            \item Why does the qubit frequency change in a SQUID loop
            \item Sweet-spot
          \end{itemize}
          \item Mention flux-noise?
          \begin{itemize}
            \item 1/f noise
            \item usually not limiting, as it is very slow
            \item This noise can be seen as occasional jumps (every few hours?) It would mean that every few hours the frequency must be recalibrated.
          \end{itemize}
        \end{itemize}

        \textbf{Figures:}
        \begin{itemize}
          \item SEM picture of SQUID loop, including flux-bias line
          \item Figure of 2D resonator scan vs DAC voltage
          \item Figure of 1D resonator fixed frequency DAC voltage scan
        \end{itemize}

      \subsection{Scanning for qubits}
        Once the preliminary measurements have been performed
        \textbf{Mention what to do if qubit is close to or far away from resonator}




      \subsection{Scanning for qubits}

    \section{Part II: Time-domain measurements}








  \chapter{Calibration routines}


  \chapter{Randomized benchmarking}

\begin{appendices}

  \chapter{Additional notes}
    \section{Qubit characterization}
      \subsection{Part I: Continuous-wave measurements}
        \subsubsection{Powersweep}
          \begin{itemize}
            \item If the resonator seems to disappear after the transition from dressed frequency to bare frequency (or vice versa), this is likely due to the qubit being extremely close to the resonator, and so the frequency shift is large.
          \end{itemize}


\end{appendices}

\end{document}